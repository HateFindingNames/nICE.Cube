%LTeX: language=DE-de
\chapter{Zielformulierung}
	%
	Hygienisches, schockabsorbierendes und handliches Transportgefäß mit autarker Energieversorgung mit Temperaturregelung für empfindliche
	Medikamente bei höchstens \SI{-20}{\degree} über mindestens \SI{48}{h} hinweg. Die Bauweise ist kompakt und ermöglicht durch sein
	Leergewicht von \(\leq \SI{20}{kg}\) die Handhabung einzelner Personen.
	%
	\section{Pflichtenheft}
		\begin{table}[h]
			\centering
			\caption{Liste unbedingt zu erfüllender Eigenschaften.}
			\begin{tabular}{@{}p{.49\textwidth}p{.49\textwidth}@{}}
				\toprule
				\textbf{Festanforderungen} 						& \textbf{Erläuterung} \\
				\midrule
				Hygienisch										& Das Material muss Kontamination durch Verkeimung
																oder Ähnlichem weitestgehend unterbinden. Die Geometrie muss eine einfache und gründliche Reinigung ermöglichen.\\
				Thermisch geregelt								& Die Innentemperatur muss einstellbar sein und einem Regelkreis unterliegen.\\
				Thermische Isolierung							& \\
				Autarke Energieversorgung						& Erneuerbare Energien nutzend muss das Gerät über eine eigene Spannungsversorgung verfügen.\\
				Integrierter Energiespeicher					& Im Offline-Betrieb und bei Ausfall der eigenen Spannungsversorgung muss das Produkt betriebsfähig bleiben.\\
				\textit{Cradle to cradle}						& Sämtliche verbauten Komponenten müssen einem biologischen oder technischen Kreislauf zurückgeführt werden können.\\
				Kompakte und leichte Bauweise 					& Das Produkt muss beladen von einer Einzelperson transportiert werden können.\\
				Schockfreie Lagerung des Inhalts 				& Der Nutzraum muss von starken Erschütterungen entkoppelt sein.\\
				Alarmierung bei kritischer Temperatur			& Die Nutzenden müssen über kritische Temperaturzustände informiert werden.\\
				Alarmierung bei kritischem Akkustand			& Die Nutzenden müssen über kritische Energieversorgung informiert werden.\\
				\bottomrule
			\end{tabular}
		\end{table}
		\newpage
		\begin{table}[h]
			\centering
			\caption{Liste nicht zu unter- oder überschreitender Parameter.}
			\begin{tabular}{@{}p{.49\textwidth}p{.49\textwidth}@{}}
				\toprule
				\textbf{Mindestanforderungen} 					& \textbf{Erläuterung} \\
				\midrule
				Kühlung bis \(\leq \SI{(-20 \pm 1)}{\celsius}\)	& \\
				Temperaturstabil \(\geq \SI{48}{h}\) 			& \\
				Gewicht \(\leq \SI{20}{kg}\)					& \\
				\bottomrule
			\end{tabular}
		\end{table}
		\newpage
		\begin{table}[h]
			\centering
			\caption{Liste von Eigenschaften, die das Produkt im Rahmen der Machbarkeit weiterhin aufweisen könnte.}
			\begin{tabular}{@{}p{.49\textwidth}p{.49\textwidth}@{}}
				\toprule
				\textbf{Wunschanforderungen} 											& \textbf{Erläuterung} \\
				\midrule
				Innenbeleuchtung 												& Bei schlechter Außenbeleuchtung sollte der Inhalt dennoch gut erkennbar sein.\\
				Statusanzeige													& Die Nutzenden sollen sich jederzeit über den Status des Produktes -- Temperatur,
																				Leistungsaufnahme und -abgabe, Restkapazität -- informieren können.\\
				Modularer Aufbau												& Einzelne funktionale Komponenten sollen leicht austauschbar sein.\\
				Verbundbetrieb													& Mehrere der Produkte sollen im Verbund betrieben werden können. Hierzu zählt Stützbetrieb einzelner Geräte durch andere, sowie
																				gesammelte Statusanzeige.\\
				Kompatibilität verschiedener ext. Spannungsversorgungen 		& Das Gerät soll an möglichst vielen verschiedenen externen Versorgungsspannungen betrieben werden können \\
				\bottomrule
			\end{tabular}
		\end{table}