%ltex: language=DE
\chapter{Überschlagsrechnung Energiebedarf}
		\emph{Annahme}:
		\SI{20}{L} Innenraum; zu 80\% gefüllt mit Wasser ($ \SI{16}{L}\widehat{=}\SI{16}{kg} $)\par\medskip
		\emph{Betrachtung:}
		Energien beim Herunterkühlen von Wasser bei RT ($ 20\SIUnitSymbolCelsius=\SI{293}{K} $) auf $ -20\SIUnitSymbolCelsius=\SI{253}{K} $\\
		
		\underline{Pro kg umgewandelte Wärmeenergie = Änderung der spezifischen Enthalpie:}\par
		Für \SI{20}{\celsius} $\rightarrow$ \SI{0}{\celsius}:
				\begin{align} 
					\Delta h_1 &= \frac{\Delta Q_1}{m} = c_{Wasser} \cdot (T_0-T_{20}) = \SI{4,19}{\frac{kJ}{kg \cdot K}} \cdot (\SI{273}{K}-\SI{293}{K}) \nonumber\\
					&= \SI{-83}{\frac{kJ}{kg}}
				\end{align}
		Zum Erstarren:
			\begin{align}
				\Delta h_2 &= h_{erstarr,Wasser} = -h_{fus,Wasser} \nonumber \\
				&= \SI{-334}{\frac{kJ}{kg}}
			\end{align}
		Für \SI{0}{\celsius} $\rightarrow$ \SI{-20}{\celsius}:
		\begin{align} 
			\Delta h_3 &= c_{Eis} \cdot (T_{-20}-T_0) = \SI{2,0}{\frac{kJ}{kg \cdot K}} \cdot (\SI{253}{K}-\SI{273}{K}) \nonumber \\
			&= \SI{-40}{\frac{kJ}{kg}}
		\end{align}

		\begin{align}
			\Delta h_{ges} 				&= \Delta h_1 + \Delta h_2 + \Delta h3 = \SI{-83,8}{\frac{kJ}{kg}} - \SI{334}{\frac{kJ}{kg}} - \SI{40}{\frac{kJ}{kg}} = \underline{\SI{-457,8}{\frac{kJ}{kg}}} \nonumber \\
			\Rightarrow \Delta Q{ges} 	&= \Delta h_{ges} \cdot m = \SI{-457,8}{\frac{kJ}{kg}} \cdot \SI{16}{kg} \nonumber \\
										&= \SI{-7324,8}{kJ} \approx \underline{\underline{\SI{-7,3}{MJ}}}
		\end{align}
		
		\underline{Benötigte Energie, um eine Temperaturdifferenz aufrechtzuerhalten:}
		\begin{align}
			\frac{\Delta Q}{\Delta t} = \dot{Q} = k \cdot A \cdot \Delta T = \SI{0,54}{\frac{W}{m^2 \cdot K}} \cdot (\SI{0,05}{m} \cdot {\SI{0,3}{m}}) \cdot (\SI{-40}{K}) = \underline{\SI{-0,324}{W}}
		\end{align}
		\begin{align}
			\Rightarrow \Delta Q &= \dot{Q} \cdot \Delta t = \SI{-0,324}{W} \cdot \SI{48}{h} \cdot \SI{3600}{\frac{s}{h}} \nonumber \\
			&= \SI{-55987,2}{J} \approx \underline{\underline{\SI{-56}{kJ}}}
		\end{align}