%LTeX: language=DE
\chapter{Materialien und Regularien}
	\section{Normen und Regularien}
		Dem \textsc{Bundesinstitut für Arzneimittel und Medizinprodukte} zu Folge ist \enquote{ein Absehen von der Genehmigung einer klinischen Prüfung} \cite{genehmigungspflicht.BfArM}
		möglich, wenn das Produkt etwa den Medizinprodukten der Klasse I zuzuordnen ist \cite{MPG}. Die Klassifizierung von medizinischen
		Produkten wiederum ist durch die \textit{Richtlinie 93/42/EWG} geregelt. In \textit{Anhang IX} ist zu lesen:
		\begin{quote}
			\enquote{Alle nicht invasiven Produkte gehören zur Klasse I [..]} \cite{directive.93-2-EC.medizinprodukte.2007}
		\end{quote}
		mit Ausnahme derer nicht-invasiven Produkte, die in folgenden Punkten aufgeführten Regelungen Anwendung finden.
		Es ist anzunehmen, dass dies hier \underline{nicht} der Fall ist.\par\medskip
	
		Dennoch handelt es sich um ein Gerät, dass, um für den europäischen Markt zugelassen werden zu können eine \underline{CE-Kennzeichnung}
		benötigt und somit die damit verbundenen Auflagen erfüllen muss. Die relevante Norm scheint hier \textsc{DIN EN 60601}.\par
		Weiter wird eine \underline{VDE-Kennzeichnung} benötigt, da es sich um ein in einer Form elektrisch betriebenes Gerät handelt.
		Dies wird um so relevanter, wenn ein Betrieb an Netzspannung möglich oder vorgesehen ist.

	\section{Auswahl Dämmmaterialien}
		\begin{table}[h]
			\centering
			\caption{Übersicht möglicher Materialien zur thermischen Dämmung.}
			\begin{threeparttable}
				\begin{tabular}{@{}lrrrr@{}}
					\toprule
					(Handels-)Name																& \(\lambda / \frac{mW}{m \cdot K}\)	& \(\rho / \frac{mg}{cm^3}\)	& E-Modul / \(kPa\) 				& Druckfestigkeit / \(kPa\) \\
					\midrule
					Luft 																		&\(25,7\)						&\(1,2041\)						&--									&-- \\
					&&&&\\
					Graphen/Silizium Aerogel \cite{silica.graphene.aerogel.Lei.2017} 			& \(7,3 \text{ -- } 9\)			&								&\((2,4 \text{ -- } 4)\cdot 10^3\)	&\\
					Biofoam (Algen) \cite{Biofoam2.Morrison.1994}								&								&30								&									&\\
					Biobasiertes PU \cite{Biobased.PU.HuangX.QiJ.DeHoopC.XieJ.andChenY.2017}	&\(33\)							&\(18,5\)						&\(176,7\)							&\(15,4\)\\
					Air laid feather fibre \cite{air.laid.feather.fibre.Zhao.2020} 				& \(30\)\tnote{1} 		&								&									&\(> 30\)\tnote{2}\\
					&&&&\\
					Vakuumdämmplatte \cite{Vakuumplate.Nagarajan.2013}							&\(37 \text{ -- } 31\)			&\(16,3 \text{ -- } 59,4\)		&									&\\
					\bottomrule
				\end{tabular}
				\begin{tablenotes}
					\footnotesize
					\item[1] Bei \SI{-10}{\celsius}
					\item[2] Nach 10 minütigem Eintauchen in flüssigen Stickstoff (\SI{-195}{\celsius})
				\end{tablenotes}
			\end{threeparttable}
		\end{table}
		%
