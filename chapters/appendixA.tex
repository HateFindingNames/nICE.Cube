%ltex: language=de-de
\chapter{Anhang A}
	\section{Pflichtenheft}
		\begin{table}[h]
			\centering
			\caption{Liste unbedingt zu erfüllender Eigenschaften.}
			\begin{tabular}{@{}p{.5\textwidth}p{.5\textwidth}@{}}
				\toprule
				\textbf{Festanforderungen} 						& \textbf{Erläuterung} \\
				\midrule
				Hygienisch										& Das Material muss Kontamination durch Verkeimung
				oder Ähnlichem weitestgehend unterbinden. Die Geometrie muss eine einfache und gründliche Reinigung ermöglichen.\\
				Thermisch geregelt								& Die Innentemperatur muss einstellbar sein und einem Regelkreis unterliegen.\\
				Thermische Isolierung							& Die innere Temperatur muss weitgehend stabil gehalten werden. \\
				Autarke Energieversorgung						& Erneuerbare Energien nutzend muss das Gerät über eine eigene Spannungsversorgung verfügen.\\
				Integrierter Energiespeicher					& Im Offline-Betrieb und bei Ausfall der eigenen Spannungsversorgung muss das Produkt betriebsfähig bleiben.\\
				\textit{Cradle-to-cradle}						& Sämtliche verbauten Komponenten müssen einem biologischen oder technischen Kreislauf zurückgeführt werden können.\\
				Kompakte und leichte Bauweise 					& Das Produkt muss beladen von einer Einzelperson transportiert werden können.\\
				Schockfreie Lagerung des Inhalts 				& Der Nutzraum muss von starken Erschütterungen entkoppelt sein.\\
				Alarmierung bei kritischer Temperatur			& Die Nutzenden müssen über kritische Temperaturzustände informiert werden.\\
				Alarmierung bei kritischem Akkustand			& Die Nutzenden müssen über kritische Energieversorgung informiert werden.\\
				\bottomrule
			\end{tabular}
		\end{table}
		% \newpage
		\begin{table}[h]
			\centering
			\caption{Liste nicht zu unter- oder überschreitender Parameter.}
			\begin{tabular}{@{}p{.5\textwidth}p{.5\textwidth}@{}}
				\toprule
				\textbf{Mindestanforderungen} 					& \textbf{Erläuterung} \\
				\midrule
				Kühlung \( \leq \SI{(-20 \pm 1)}{\celsius} \)	& Der Innenraum muss auf mindestens \SI{-20}{\celsius} heruntergekühlt werden können. \\
				Temperaturstabil \(\geq \SI{48}{\hour}\) 			& Die Temperatur muss min. \SI{48}{\hour} lang gehalten werden. \\
				Gewicht \(\leq \SI{20}{\kilo\gram}\)					& Die Box soll ein Leergewicht von max. \SI{20}{\kilo\gram} besitzen. \\
				\bottomrule
			\end{tabular}
		\end{table}
		% \newpage
		\begin{table}[h]
			\centering
			\caption{Liste von Eigenschaften, die das Produkt im Rahmen der Machbarkeit weiterhin aufweisen könnte.}
			\begin{tabular}{@{}p{.5\textwidth}p{.5\textwidth}@{}}
				\toprule
				\textbf{Wunschanforderungen} 											& \textbf{Erläuterung}\\
				\midrule
				Innenbeleuchtung 												& Bei schlechter Außenbeleuchtung sollte der Inhalt dennoch gut erkennbar sein.\\
				Statusanzeige													& Die Nutzenden sollen sich jederzeit über den Status des Produktes -- Temperatur,
				Leistungsaufnahme und -abgabe, Restkapazität -- informieren können.\\
				Modularer Aufbau												& Einzelne funktionale Komponenten sollen leicht austauschbar sein.\\
				Verbundbetrieb													& Mehrere der Produkte sollen im Verbund betrieben werden können. Hierzu zählt Stützbetrieb einzelner Geräte durch andere, sowie gesammelte Statusanzeige.\\
				Kompatibilität verschiedener ext. Spannungsversorgungen 		& Das Gerät soll an möglichst vielen verschiedenen externen Versorgungsspannungen betrieben werden können.\\
				\bottomrule
			\end{tabular}
		\end{table}

	\section{Marktforschung}
	Es wurden Marktforschungen betrieben, um eine Übersicht der konkurrierenden Kühlboxen zu erhalten.\par\medskip

	\textbf{T0022 FDN von} \href{https://www.eberspaecher-klima.de/fileadmin/data/corporatesite/pdf/de/4_air_conditioning/gp/fh_gp_kuehlcontainer_de.pdf}{\textsc{Eberspächer}}

	\begin{itemize}
		\item Temperatur von \SI{-24}{\celsius}
		\item Spannung \SI{12}{\volt} / \SI{24}{\volt} DC
		\item \SI{(65-70)}{\milli\metre} PU Wandausschämung
		\item Kältemittel R134a (Tetrafluorethan)
		\item Gewicht von \SI{20,5}{\kilo\gram}
		\item Außenabmessung 375 x 585 x 480 (BxTxH, Einheiten in mm)
	\end{itemize}

	\textbf{ICY 20 von} \href{https://www.frigolab.eu/gb/dometic-portable-freezers/67-icy-20.html#/47-normal_or_heated_refrigerator-heated_18c40c}{\textsc{FrigoLab Cold Technology}}

	\begin{itemize}
		\item Temperaturen von \SI{-18}{\celsius} bis \SI{10}{\celsius}.
		\item Außenabmessung \SI{(515 x 345 x 425)}{mm} (BxTxH).
		\item Digitales Display.
		\item Kühlung durch einen Kompressor.
		\item Kältemittel R134a (Tetrafluorethan).
		\item Versorgung von \SI{12}{\volt} / \SI{24}{\volt} DC oder \SI{230}{\volt} AC.
		\item Gewicht von \SI{18}{\kilo\gram}.
		\item Resistent gegen Schock und mechanische Belastungen.
		\item CFC- und HCF-freie Polyurethanschaum.
	\end{itemize}

	\textbf{TC 702 von} \href{https://www.tritec-klima.de/datenblaetter/de/kaelte/portable/TD-TC702.pdf}{\textsc{Tritec}}

	\begin{itemize}
		\item Temperaturen von \SI{-24}{\celsius} bis \SI{10}{\celsius}.
		\item Außenabmessung \SI{(375 x 585 x 480)}{\milli\metre} (B x T x H).
		\item Durch Autobatterie mit \SI{12}{\volt} / \SI{24}{\volt} DC betrieben.
		\item Umschalter für Netzbetrieb (\SI{230}{\volt} AC).
		\item Stabiler und stoßfester UVA-beständiger Kunststoff.
		\item Deckel mit spezieller Verriegelung.
		\item Dämmung mit FCKW-feiem PU-Hartschaum von \SI{80}{\milli\metre} Dicke.
		\item Reinigungsmittelresistent.
		\item Alarmmeldungen bei möglichen Fehlern des Gerätes.
		\item Kompressor mit einem R52A Kältemittelgemisch.
	\end{itemize}

	\textbf{T0022/T0032 von} \href{https://coldtainerusa.com/wp-content/uploads/2020/03/Product_Info_T0022-T0032_US_ColdtainerUSA-1.pdf}{\textsc{Coldtainer}}

	\begin{itemize}
		\item Temperatur von \SI{-24}{\celsius} bis \SI{40}{\celsius}
		\item Spannungsversorgung 12 / \SI{24}{\volt} DC %volt of direct current bzw. Gleichstrom
		\item \SI{(65-70)}{\milli\metre} PU-Wandausschämung
		\item Kältemittel R134a (Tetrafluorethan)
		\item Gewicht von \SI{20,5}{\kilo\gram}
		\item \(\approx \SI{23}{L}\) Kapazität
	\end{itemize}

	\section{Materialien und Regularien}
		\subsection{Normen und Regularien}
			Dem \textsc{Bundesinstitut für Arzneimittel und Medizinprodukte} zu Folge ist \enquote{ein Absehen von der Genehmigung einer klinischen Prüfung} \cite{genehmigungspflicht.BfArM}
			möglich, wenn das Produkt etwa den Medizinprodukten der Klasse I zuzuordnen ist \cite{MPG}. Die Klassifizierung von medizinischen
			Produkten wiederum ist durch die \textit{Richtlinie 93/42/EWG} geregelt. In \textit{Anhang IX} ist zu lesen:
			\begin{quote}
				\enquote{Alle nicht invasiven Produkte gehören zur Klasse I [..]} \cite{directive.93-2-EC.medizinprodukte.2007}
			\end{quote}
			mit Ausnahme derer nicht-invasiven Produkte, die in folgenden Punkten aufgeführten Regelungen Anwendung finden.
			Es ist anzunehmen, dass dies hier \underline{nicht} der Fall ist.\par\medskip

			Dennoch handelt es sich um ein Gerät, dass, um für den europäischen Markt zugelassen werden zu können eine \underline{CE-Kennzeichnung}
			benötigt und somit die damit verbundenen Auflagen erfüllen muss. Die relevante Norm scheint hier \textsc{DIN EN 60601}.\par
			Weiter wird eine \underline{VDE-Kennzeichnung} benötigt, da es sich um ein in einer Form elektrisch betriebenes Gerät handelt.
			Dies wird um so relevanter, wenn ein Betrieb an Netzspannung möglich oder vorgesehen ist.

		\subsection{Auswahl Dämmmaterialien}
			\begin{table}[H]
				\centering
				\caption{Übersicht möglicher Materialien zur thermischen Dämmung.}
				\begin{tabular}{@{}lrrrr@{}}
					\toprule
					(Handels-)Name																& \(\lambda / \frac{mW}{m \cdot K}\)	& \(\rho / \frac{mg}{cm^3}\)	& E-Modul / \(kPa\) 				& Druckfestigk. / \(kPa\) \\
					\midrule
					Luft 																		&\(25,7\)						&\(1,2041\)						&--									&-- \\
					&&&&\\
					Graphen/Silizium Aerogel \cite{silica.graphene.aerogel.Lei.2017} 			& \(7,3\) -- \(9\)			&								&\((2,4\) -- \(4\))\(\cdot 10^3\)	&\\
					Biofoam (Algen) \cite{Biofoam2.Morrison.1994}								&								&30								&									&\\
					Biobasiertes PU \cite{Biobased.PU.HuangX.QiJ.DeHoopC.XieJ.andChenY.2017}	&\(33\)							&\(18,5\)						&\(176,7\)							&\(15,4\)\\
					Air laid feather fibre \cite{air.laid.feather.fibre.Zhao.2020} 				& \(30^1\)		&								&									& \( > 30^2 \) \\
					&&&&\\
					Vakuumdämmplatte \cite{Vakuumplate.Nagarajan.2013}							&\(37\) -- \(31\)			&\(16,3\) -- \(59,4\)		&									&\\
					\bottomrule
				\end{tabular}
			\end{table}
			\( ^1 \)Bei \SI{-10}{\celsius}\\
			\( ^2 \)Nach 10 minütigem Eintauchen in flüssigen Stickstoff (\SI{-195}{\celsius})

	\section{Überschlagsrechnung Energiebedarf}
		\textit{Annahme}:
		\SI{20}{\litre} Innenraum; zu \SI{80}{\percent} gefüllt mit Wasser (\( \SI{16}{\litre} \widehat{=} \SI{16}{\kilo\gram} \))\par\medskip
		\textit{Betrachtung:}
		Energien beim Herunterkühlen von Wasser bei RT (\( \SI{20}{\celsius} = \SI{293}{\kelvin} \)) auf \( \SI{-20}{\celsius} = \SI{253}{\kelvin} \)\par\medskip

		\underline{Je kg umgewandelte Wärmeenergie \(=\) Änderung der spezifischen Enthalpie:}\par\medskip
		Für \SI{20}{\celsius} \(\rightarrow\) \SI{0}{\celsius}:
		\begin{align}
			\Delta h_1	&= \frac{\Delta Q_1}{m} = c_{Wasser} \cdot \left(T_0 - T_{20}\right) = \SI{4,19}{\kilo\joule\per\kilo\gram\kelvin}\\% \cdot \left(\SI{273}{\kelvin} - \SI{293}{\kelvin}\right) \nonumber\\
						&= \SI{-83}{\kilo\joule\per\kilo\gram}
		\end{align}
		Zum Erstarren:
		\begin{align}
			\Delta h_2	&= h_{erstarr,Wasser} = -h_{fus,Wasser} \nonumber \\
						&= \SI{-334}{\kilo\joule\per\kilo\gram}
		\end{align}
		Für \SI{0}{\celsius} \(\rightarrow\) \SI{-20}{\celsius}:
		\begin{align}
			\Delta h_3	&= c_{Eis} \cdot (T_{-20}-T_0) = \SI{2,0}{\kilo\joule\per\kilo\gram\kelvin} \cdot (\SI{253}{\kelvin}-\SI{273}{\kelvin}) \nonumber \\
						&= \SI{-40}{\kilo\joule\per\kilo\gram}
		\end{align}

		\begin{align}
			\Delta h_{ges} 				&= \Delta h_1 + \Delta h_2 + \Delta h3 = \SI{-83,8}{\kilo\joule\per\kilo\gram} - \SI{334}{\kilo\joule\per\kilo\gram} - \SI{40}{\kilo\joule\per\kilo\gram} = \underline{\SI{-457,8}{\kilo\joule\per\kilo\gram}} \nonumber \\
			\Rightarrow \Delta Q{ges} 	&= \Delta h_{ges} \cdot m = \SI{-457,8}{\kilo\joule\per\kilo\gram} \cdot \SI{16}{\kilo\gram} \nonumber \\
										&= \SI{-7324,8}{\kilo\joule} \approx \underline{\underline{\SI{-7,3}{\mega\joule}}}
		\end{align}

		\underline{Benötigte Energie, um eine Temperaturdifferenz aufrechtzuerhalten:}\par\smallskip
		Mit einer Schichtdicke des thermischen Isoliermaterials von \(l = \SI{0,05}{\metre}\), der Gesamtfläche \(A\) des Kühlraums von \(\approx \SI{0,208}{\metre\squared}\)
		und einer thermischen Leitfähigkeit von \(\lambda = \SI{7,3}{\mW\per\metre\kelvin}\) lässt sich der thermische Widerstand zu
		\(R_{th} \approx \SI{33}{\kelvin\per\watt}\) abschätzen.
		\begin{equation}
			R_{th} 	= \frac{l}{\lambda \cdot A} = \frac{\SI{0,05}{\metre}}{\SI{7,3}{\mW\per\metre\kelvin} \cdot \SI{0,208}{\metre\squared}} \approx \SI{33}{\kelvin\per\watt}
		\end{equation}

		\begin{align}
			\frac{\Delta Q}{\Delta t} = \dot{Q} = k \cdot A \cdot \Delta T = \SI{0,54}{\watt\metre\squared\per\kelvin} \cdot (\SI{0,05}{\metre} \cdot \SI{0,3}{\metre}) \cdot (\SI{-40}{\kelvin}) = \underline{\SI{-0,324}{\watt}}
		\end{align}
		\begin{align}
			\Rightarrow \Delta Q	&= \dot{Q} \cdot \Delta t = \SI{-0,324}{\watt} \cdot \SI{48}{\hour} \cdot \SI{3600}{\frac{\second}{\hour}} \nonumber \\
									&= \SI{-55987,2}{\joule} \approx \underline{\underline{\SI{-56}{\kilo\joule}}}
		\end{align}

		\begin{align*}
			\dot{Q} = P	&= \frac{(T_{aussen} - T_{innen})}{R_{th}} \\
						&= \frac{\SI{313,15}{\kelvin} - \SI{253,15}{\kelvin}}{\SI{33}{\kelvin\per\watt}} \\
						&\approx \SI{1,81}{\watt} \\
			\Rightarrow &\approx \SI{87,27}{\watt\hour} \qquad \text{(Für \SI{48}{\hour})}
		\end{align*}
		%
		Bei einer Außentemperatur von \SI{40}{\celsius} und einer Anfangsinnentemperatur von \SI{-20}{\celsius} mit der Annahme, dass \SI{33}{\percent} des Nutzvolumens von \SI{16,72}{\litre} mit Wasser und
		\SI{66}{\percent} mit Luft gefüllt sind, wird von einer Dauer von \(t_{krit} = \SI{}{\hour}\) ausgegangen, bis sich der Innenraum um \SI{2}{\kelvin} erwärmt hat.
		\begin{align}
			t_{krit}	&= \frac{\Delta \vartheta}{\dot{Q}} \cdot \left(0,66 \cdot \rho_{aq}V \cdot c_{aq}\right) + \left(0,33 \cdot \rho_{luft}V \cdot c_{luft} \right)\nonumber \\
						&= \frac{\SI{2}{\kelvin}}{\SI{1,81}{\watt}} \cdot \left( 0,66 \cdot \SI{1000}{\kilo\gram\per\metre\cubed} \cdot \SI{0,206}{\metre\cubed} \cdot \SI{4200}{\joule\kelvin\per\kilo\gram}\right)\nonumber \\
						&+ \left(  0,33 \cdot \SI{1,225}{\kilo\gram\per\metre\cubed} \cdot \SI{0,206}{\metre\cubed} \cdot \SI{1005}{\joule\kelvin\per\kilo\gram} \right)\nonumber \\
						&\approx 
		\end{align}
		\begin{align}
			\Delta \vartheta	&= \frac{\Delta Q}{0,66 \cdot \rho_{aq}V \cdot c_{aq}}\nonumber\\
								&= \frac{\SI{1,81}{\watt} \cdot \SI{5}{\hour} \cdot \SI{3600}{\second \per \hour} }{0,66 \cdot \SI{997}{\kilogram \per \metre\cubed} \cdot \SI{ (16,72 \cdot 10^{-3}) }{ \metre\cubed } \cdot \SI{4200}{\joule \per \kilogram \kelvin}}\nonumber\\
								&\approx \SI{0,7}{\kelvin}
		\end{align}

		%=============== ???REDUNDANT??? ==================
	% \section{Realisierung}
	% 	\begin{align*}
	% 		\dot{Q} = P &= \frac{\lambda \cdot A \cdot (T_{aussen} - T_{innen})}{d} \\
	% 					&= \frac{\SI{0,009}{\watt\per\metre\kelvin} \cdot \SI{431857,65 \cdot 10^{-6}}{\metre\squared} \cdot (\SI{313,15}{\kelvin} - \SI{253,15}{\kelvin})}{\SI{50 \cdot 10^{-3}}{\metre}} \\
	% 					&\approx \SI{4,66}{\watt} \\
	% 		\Rightarrow &\approx \SI{223,88}{\watt\hour} \qquad \text{(Für \SI{48}{\hour})}
	% 	\end{align*}
	% 	%
	% 	\begin{align*}
	% 		\Delta T 	&= \frac{\Delta Q}{m \cdot c_{H_2O,liq}} \\
	% 					&= \frac{\SI{4,66}{\watt} \cdot \SI{5}{h}}{\SI{10}{\kilo\gram} \cdot \SI{4,2 \cdot 10^{3}}{\joule\per\kilo\gram\kelvin}} \cdot \SI{3600}{\second\per\hour} \\
	% 					&\approx \SI{2}{\kelvin}
	% 	\end{align*}
	%=================================================
	
	\section{Dämmung}
		